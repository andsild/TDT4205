\section*{Task 1.1}
First:\
\begin{definition}[Syntax Directed Definitions]
    A context free grammar where grammar symbols have semantic operations, named attributes.
\end{definition}
\begin{definition}[Synthesized attribute]
    A value which is only given from a given non-terminal or any of its children.
\end{definition}
\begin{definition}[Inherited attribute]
    A value for a non-terminal in a parse tree which is given from either a parent or sibling node.
\end{definition}

An S-attributed definition is an SDD where all attributes are synthesized.
An L-attributed definition is an SDD where attributes can be either synthesized or 
inherited. When using inherited attributes, there may be pitfalls that make interpretation difficult.  Typically, to do translation of L-attributed SDDs by traversing parse trees from left to right in a postorder, top-down fashion. 
To avoid trouble, we apply the following constraints to inherited attributes:
\begin{enumerate}
    \item Inheritance must be to the left of any given node. I.e.\ for a given
    production rule $A \rightarrow X_{1}X_{2}\dots$, any given $X_{i}$ can only
    have its value computed by using values from $X_{j}$, for any $j < i$.
    \item There cannot be cyclic definition, e.g.\ the following set is invalid:
    $X \rightarrow A, A \rightarrow X$.
\end{enumerate}

\noindent Since synthesized attributes do not yield the same traversal problems, S-attributed
defintions do not need extra constraints. S-attributed SDD can thusly be parsed bottom-up without much trouble. Since all s-attributed SDD's can also be valid as L-attributed grammars, they can also be parse the same way as an L-attributed grammar.


\section*{Task 1.2}
I define the following rules:
\begin{enumerate}
    \item $P(a,b) = a > b \wedge b:t \implies a:t$
    \item $Q(a,b) = a : b \wedge b:t \implies a:t$
    \item $R(a,b,c) = a:bool \wedge b:t \wedge c:t \implies ((A) b : c) : t$
\end{enumerate}

\begin{tikzpicture}[sibling distance=15em,
  every node/.style = {shape=rectangle,
    align=center,
    top color=white, bottom color=white!20}]]
  \node {$(x > 2) : y : 3.14$}
    child { node {$x > 2$}
        child { node {$2 : int$ }
            child { node {$P(x,2) \implies (x > 2): bool$}
                } } }
    child { node {$y : 3.14$}
      child { node {$3.14:float$}
        child { node {$Q(y, 3.14) \implies y : float$} }
    } }
    child { node {$R( (x > 2), y, 3.14) \implies (x > 2) y : 3.14 : float$}
    };
\end{tikzpicture}

\section*{Task 2}
Okay. My output seems to match the .table.correct files for all inputs.
The only difference is the ordering (assumably because I use a different hashing scheme than you did).

